\documentclass[11pt]{article}

\usepackage{a4wide}
\usepackage[T1]{fontenc}
\usepackage{xcolor}
\usepackage{ulem}					% zum Unterstreichen

\usepackage[utf8]{inputenc}
\usepackage{textcomp}				% für Eurozeichen etc.
\usepackage[french]{babel} 		% Für Deutsche " etc.
\usepackage{xspace} 				% Setzt Leerzeichen richtig
\usepackage[none]{hyphenat} 		% Schaltet Silbentrennung aus
\sloppy								% Erzwingt Blocksatz (aber größerer Wortabstand)

\usepackage{tabularx}				% modifizierte Tabellen
\usepackage{graphicx}				% Grafiken einbinden
\usepackage{enumitem}				% für Aufzähluingen, Listen

\usepackage{amsmath}				% Schickes Mathe-Krams
\usepackage{amssymb}				% Symbolik (Mengensymbole, etc.)
\usepackage{amstext}				% erlaubt es, im Mathemodus mit den Befehl \text{..} für Fließtext zu schreiben
\usepackage{amsthm}					% für beweise
\usepackage{mathtools}				% erlaubt das Schreiben von Wörtern über/unter Pfeilen
\usepackage{polynom}				% berechnet Polynomdivision automatisch
\usepackage{esvect}					% schicke Vektorpfeile "\vv{..}"
\usepackage{nicefrac}				% Für Quotientenraum

\usepackage{cancel}					% Für Durchstreichungen
\usepackage{arydshln}				% Für gestrichelte Linien

\usepackage{listings}

\definecolor{mygreen}{rgb}{0,0.6,0}
\definecolor{mygray}{rgb}{0.5,0.5,0.5}
\definecolor{mymauve}{rgb}{0.58,0,0.82}

\lstset{
commentstyle=\color{mygreen},
keywordstyle=\color{blue},
numberstyle=\tiny\color{red}}

\setlength{\parindent}{0em} 		% Einrückung der ersten Zeile in einem Absatz
\setlength{\parskip}{1em} 			% Abstand eines Absatz zum nächsten

% EIGENE KOMMANDOS
\newcommand{\define}{\textbf{\textit{Definition: }}}
\newcommand{\inv}[1]{\overline{#1}}
\newcommand{\wo}{\textbackslash}
\newcommand{\zbN}{\mathbb{N}}
\newcommand{\zbZ}{\mathbb{Z}}
\newcommand{\zbQ}{\mathbb{Q}}
\newcommand{\zbR}{\mathbb{R}}
\newcommand{\zbC}{\mathbb{C}}
\newcommand{\metaInfo}[4]{
	\begin{tabularx}{\textwidth}{rXrl}
	\hline \\
	\textbf{Groupe:}	& #1 &  \textbf{Rendu le:}   & #2 \\
	\\ \hline
	\end{tabularx}
}
\newcommand{\heading}[2]{
	\begin{center}
	\section*{#1}
	\section*{#2}
	\end{center}
}
\renewcommand\thesection{\Roman{section}}


\begin{document}
\heading{Mini-Projet: Entrepôts de Données}{AirBNB}
\metaInfo{Leonie Berwanger, Tahani Qattan,\newline Leila Rekkab, Khady Ajna Thiam}{\today}

\section{}
\section{Considérations préliminaires}
\subsection{Quelles sont les actions/opérations importantes pour la plateforme de locations de vacances AirBNB?}
\begin{itemize}
	\item Réservation d'un logement
	\item Annulation
	\item Recherche sur le plateforme
	\item Rajout d'un nouveau logement/ d'un nouveau compte d'usageur
\end{itemize}
\subsection{Pour chaque opération: trois traitements possibles permettant d’aider à la prise de décision sur le sujet.}
\subsubsection*{Réservation d'un logement}
\begin{itemize}
	\item Est-ce qu'il est plus raisonnable pour les hôtes d'offrir une réservation directe sans confirmation obligatoire de leur côté?
	\item Quels sont les prix moyens pour les différents types de logements par ville (par saison)?
	\item Est-ce que la réduction offerte lorsqu'on invite d'autres utilisateurs est utile à apporter plus de réservations à la plateforme?
\end{itemize}
\subsubsection*{Annulation}
\begin{itemize}
	\item Est-ce que les mesures à pénaliser les annulations ont un effet découvrable?
	\item  Quels sont les paramètres qui peuvent provoquer une annulation?
	\item 
\end{itemize}
\subsection{L'ordre d'importance}


\section{Conception de l'entrepôt}
\subsection{Les deux actions les plus importantes à analyser}
\subsection{Réservation d'un logement}

\subsection{Annulation}

\subsection{Les dimensions}

\subsection{Répondre aux traitements}

\subsection{Un exemple d'instance}

\end{document}