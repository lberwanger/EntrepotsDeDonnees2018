\documentclass[11pt]{article}

\usepackage{a4wide}
\usepackage[T1]{fontenc}
\usepackage{xcolor}
\usepackage{ulem}					% zum Unterstreichen

\usepackage[utf8]{inputenc}
\usepackage{textcomp}				% für Eurozeichen etc.
\usepackage[french]{babel} 		% Für Deutsche " etc.
\usepackage{xspace} 				% Setzt Leerzeichen richtig
\usepackage[none]{hyphenat} 		% Schaltet Silbentrennung aus
\sloppy								% Erzwingt Blocksatz (aber größerer Wortabstand)

\usepackage{tabularx}				% modifizierte Tabellen
\usepackage{graphicx}				% Grafiken einbinden
\usepackage{enumitem}				% für Aufzähluingen, Listen

\usepackage{amsmath}				% Schickes Mathe-Krams
\usepackage{amssymb}				% Symbolik (Mengensymbole, etc.)
\usepackage{amstext}				% erlaubt es, im Mathemodus mit den Befehl \text{..} für Fließtext zu schreiben
\usepackage{amsthm}					% für beweise
\usepackage{mathtools}				% erlaubt das Schreiben von Wörtern über/unter Pfeilen
\usepackage{polynom}				% berechnet Polynomdivision automatisch
\usepackage{esvect}					% schicke Vektorpfeile "\vv{..}"
\usepackage{nicefrac}				% Für Quotientenraum

\usepackage{cancel}					% Für Durchstreichungen
\usepackage{arydshln}				% Für gestrichelte Linien

\usepackage{listings}

\definecolor{mygreen}{rgb}{0,0.6,0}
\definecolor{mygray}{rgb}{0.5,0.5,0.5}
\definecolor{mymauve}{rgb}{0.58,0,0.82}

\lstset{
commentstyle=\color{mygreen},
keywordstyle=\color{blue},
numberstyle=\tiny\color{red}}

\setlength{\parindent}{0em} 		% Einrückung der ersten Zeile in einem Absatz
\setlength{\parskip}{1em} 			% Abstand eines Absatz zum nächsten

% EIGENE KOMMANDOS
\newcommand{\define}{\textbf{\textit{Definition: }}}
\newcommand{\inv}[1]{\overline{#1}}
\newcommand{\wo}{\textbackslash}
\newcommand{\zbN}{\mathbb{N}}
\newcommand{\zbZ}{\mathbb{Z}}
\newcommand{\zbQ}{\mathbb{Q}}
\newcommand{\zbR}{\mathbb{R}}
\newcommand{\zbC}{\mathbb{C}}
\newcommand{\metaInfo}[4]{
	\begin{tabularx}{\textwidth}{rXrl}
	\hline \\
	\textbf{Groupe:}	& #1 &  \textbf{Rendu le:}   & #2 \\
	\\ \hline
	\end{tabularx}
}
\newcommand{\heading}[2]{
	\begin{center}
	\section*{#1}
	\section*{#2}
	\end{center}
}
\renewcommand\thesection{\Roman{section}}


\begin{document}
\heading{Mini-Projet: Entrepôts de Données}{AirBNB}
\metaInfo{Leonie Berwanger, Tahani Qattan,\newline Leila Rekkab, Khady Ajna Thiam}{23 octobre 2018}

\section{}
\section{Considérations préliminaires}
\subsection{Quelles sont les actions/opérations importantes pour la plateforme de locations de vacances AirBNB?}
\begin{itemize}
	\item Réservation d'un logement par un client
	\item Offre d'un logement par un hôte
	\item Annulation d'une réservation
\end{itemize}
\subsection{Pour chaque opération: trois traitements possibles permettant d’aider à la prise de décision sur le sujet.}
\subsubsection*{Réservation d'un logement par un client}
\begin{itemize}
	\item Le prix moyen (/nuit) des differents logements réservés par type
	\item Le prix moyen (/nuit) des differents logements réservés par ville
	\item Nombre de reservations par ville et par type de logement
\end{itemize}
\subsubsection*{Offre d'un logement par un hôte}
\begin{itemize}
	\item Le nombre de logements offerts par des \og Superhosts \fg{} par ville
	\item Le type de logement le plus offert 
	\item Le prix de base moyen par nuit proposé par type de logement offert
\end{itemize}
\subsubsection*{Annulation d'une réservation}
\begin{itemize}
	\item Les raisons d'annulations les plus fréquentes
	\item Les hôtes les plus exposés aux annulations
\end{itemize}
\subsection{L'ordre d'importance}
\begin{enumerate}
	\item Réservation d'un logement par un client: Permet d'augmenter la rentabilité de la plateforme
	\item Offre d'un logement par un Hôte : Permet d'augementer le nombre de logememnt offerts et par conséquent l'augmentation du nombre de réservations
	\item Annulation d'une réservation par un client: Permet de retirer de l'espace de stockage de la plateform les logements qui ne sont pas fiables
\end{enumerate}
\section{Conception de l'entrepôt}
\subsection{Les deux actions les plus importantes à analyser}
\begin{enumerate}
	\item Réservation d'un logement par un client
	\item Offre d'un logement par un hôte
\end{enumerate}
\subsection{Réservation d'un logement}
\begin{figure}[h]
	\centering
	\includegraphics*[width=7cm]{img/reservation.png}
	\caption{Modèle en étoile des réservations}
\end{figure}

\subsubsection*{La table des faits}
Pour enregistrer les réservations, on choisit des faits transactionnels, parce qu'on s'intéresse aux caractéristiques des réservations spécifique et il ne sert à rien de les regrouper à l'avance. La table des faits suit le modèle relationnel suivant:\\
Reservations(\\
idLogement ChaineCar,\\
idLocalisation ChaineCar,\\
idHote ChaineCar,\\
idClient ChaineCar,\\
idPeriode ChaineCar,\\
nbVoyageurs Integer,\\
prixReservation Nombre(10,2))
\subsubsection*{Les mesures}
La table des faits contient les mesures \textit{nbVoyageurs} et \textit{prixReservation}. Ces mesures sont additives, parce qu'elles sont exprimées en nombres et liées à une certaine transaction. 
\subsection{Offre d'un logement par un hôte}
Pour les logements offerts on crée une table de snapshots (une période raisonnable serait 1 jour). À la fin d'une période fixe on enregistre le nombre de logements par localisation et type de logement.
\subsubsection*{La table des faits}
Un fait \textbf{(idLoc, typeLog, idDispo, idCap, j, x)} existe pour chaque combinaison de localisation idLoc, type de logement typeLog, disponibilité idDispo, capacité maximale de voyageurs cap et jour j. La mesure x représente le nombre de logements du type typeLog, au lieu idLoc, disponibles/indisponibles, pouvant reçevoir cap voyageurs le jour j.
\subsubsection*{Les mesures}
La mesure x est semi-additive, parce qu'on peut p.e. prendre le somme de logements dans toutes les localisations et de tous les types, mais on ne peut pas additionner à travers de la dimension du temps.
\subsection{Les dimensions}
Les tables des dimensions suivent le modèle relationnel suivant:\\
Logement(rf.II.2, 12 attributs)\\
Localisation(rf.II.2, 9 attributs)\\
Client(9 attributs)\\
Hôte(9 attributs)\\
Periode(6 attributs)\\
Abstract Type;
TypeLogement(typeName ChaineCar, MotsClés Array<ChaineCar>, termesLégales Array<ChaineCar>, dateRajout Date) dans une table d'objets\\
Disponibilité(idDisponibilite ChaineCar, nomDisponibilite ChaineCar, limité Integer[0,1], dernierChangement Integer[0,..,2])\\
Capacité(idCap, description ChaineCar, nbMinVoyageurs Integer, nbMaxVoyageurs Integer)

\subsection{Répondre aux traitements}
Le schéma défini permet de répondre aux traitements.
\subsubsection*{Réservation d'un logement par un client}
\begin{itemize}
	\item Un boucle imbriqué qui calcule le prix par nuit pour les réservations,après un GROUP BY typeLog est l'application de la mesure AVG()
	\item idem, GROUP BY ville
	\item COUNT(*),GROUP BY typeLog, ville
\end{itemize}
\subsubsection*{Offred'un logement par un hôte}
\begin{itemize}
	\item 
	\item SELECT (typeLog,x) FROM FactTableOffres;
	\item 
\end{itemize}

\subsection{Un exemple d'instance}
\lstinputlisting[language=sql]{exemple.sql}
\end{document}